\section{Einleitung}
\subsection{Wer bin ich?}
Ich heisse Cédric Skwar und bin stolzes Mitglied der EIA (Elektronik, Informatik und Automation) Klasse. Ich bin 16 Jahre alt und habe mich für eine Lehre als Informatiker EFZ im Fachbereich Applikationsentwicklung entschieden. Zu meinen Hobbys zählen das Programmieren in verschiedenen Programmiersprachen sowie das Lesen von Büchern in verschiedenen Sprachen wie Englisch oder Deutsch und das Fahrradfahren.
Ich habe Zertifikate in verschiedenen Themen, wie Linux, Python und JavaScript.\\
Privatsphäre ist für mich ein sehr wichtiger Aspekt, deshalb nutze ich die Linux-Distribution "Arch Linux" und verwende hauptsächlich Open-Source-Software. Auch mein Messenger ist Open-Source und der Quellcode ist auf meinem GitHub-Profil verfügbar. Für das Projekt verwende ich die MIT-Lizenz, eine detaillierte Erläuterung dieser Lizenz finden Sie unter folgendem Link\footnote{\hyperlink{https://choosealicense.com/licenses/mit/}{https://choosealicense.com/licenses/mit/}}.
\subsection{Wie lautet mein Thema?}
Mein Projektthema ist es, einen dezentralisierten Messenger in der Programmiersprache Go zu schreiben, wobei der hauptsächliche Fokus auf dem Networking liegt. Der Messenger wird auf Linux, Android und Windows kompiliert werden. Ein weiterer Fokus, dem ich viel Wert beimesse, ist das UI/UX des fertigen Messengers.
\subsection{Warum habe ich mich für dieses Projektthema entschieden?}
Ich habe mich für dieses Thema entschieden, weil ich bereits in meiner letzten Projektarbeit ein ähnliches Projekt umsetzen wollte. Das Thema interessiert mich jedoch sehr, da es mir ermöglicht, mich mit vielen neuen Themen auseinanderzusetzen und mein Wissen in diesem Bereich zu erweitern.
\subsection{Kosten}
Ich plane, für mein gesamtes Projekt voraussichtlich keinen einzigen Rappen auszugeben, da der Editor, den ich verwenden werde, Open-Source und kostenlos ist. Auch benötige ich keine Hardware oder Server, da der Messenger dezentralisiert ist und keine zentrale Serverinfrastruktur benötigt.
\subsection{Wie ich an die nötigen Informationen kommen werde?}
Ich werde wahrscheinlich Informationen durch Web-Suchen und YouTube-Videos finden und dabei mein bestehendes Wissen nutzen
\subsection{Was erhoffe ich mir von diesem Projekt?}
Ich hoffe, dass ich mein Wissen über Dezentralisierung, Networking und UI/UX verbessern kann.
\subsection{Vorausichtliche Funktionen}
Ich plane, einen Messenger zu entwickeln, der es den Nutzern ermöglicht, Bilder und Text-Nachrichten zu senden, ohne dass eine Telefonnummer erforderlich ist. 
Der Messenger soll auch sicher und datenschutzfreundlich sein. Die Nutzerdaten sollten auf dem dezentralisierten Netzwerk verteilt werden, so dass kein einziger Nutzer alle Daten hat. Dies trägt zur Sicherheit und zum Schutz der Privatsphäre bei, da ein Angreifer nicht in der Lage ist, alle Daten durch einen einzigen Angriffspunkt zu hacken, sondern alle Datenpunkte angreifen müsste.