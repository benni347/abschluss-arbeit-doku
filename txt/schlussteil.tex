\section{Schlussteil}
\subsection{Fazit}
Mein Fazit ist, dass ich meinen Start mit dem Minimum Viable Product (MVP) zu spät begonnen habe, was zu erheblichen Verzögerungen in meiner Planung geführt hat. Für ein Messenger-MVP sollten grundlegende Funktionen wie das Senden und Empfangen von Nachrichten sowie das Starten eines Chats implementiert sein. Erst nachdem diese Funktionen umgesetzt sind, werde ich mich dem UI/UX-Design widmen. Ich erkenne nun den Wert dieser Vorgehensweise und werde sie bei zukünftigen Projekten, sei es privat, akademisch oder beruflich, berücksichtigen. Außerdem habe ich erkannt, dass ich mein Programm nicht von Grund auf neu erstellen sollte. Hätte ich zuerst mein MVP erreicht, hätte ich anschließend einfache Funktionen wie Verschlüsselung problemlos hinzufügen können. Leider habe ich in diesem Projekt den umgekehrten Ansatz gewählt.
\subsection{Zukunft des Projekts}
Mein Vorhaben ist es, auch künftig an dem Messenger zu arbeiten und sowohl zusätzliche Funktionalitäten als auch Verbesserungen für meine bereits existierende Code-Basis zu entwickeln. Es ist mein Ziel, kontinuierlich an diesem Projekt zu arbeiten, um eine fortwährende Verbesserung zu gewährleisten. Den Quellcode sowie zukünftige Änderungen können Sie in meinem Github-Repository unter folgendem Link einsehen: \href{https://github.com/benni347/messenger}{https://github.com/benni347/messenger}.
\subsection{Danksagung und Mitwirkung"}
Mein Dank gilt meinem Vater, der mir insbesondere bei der Erstellung dieser Dokumentation assistierte, indem er die Textlänge überprüfte und Vorschläge für Modifikationen unterbreitete, die den Text verständlicher gestalten könnten.