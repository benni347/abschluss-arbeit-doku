\section{Schlussteil}
\subsection{Fazit}
Mein persönliches Fazit ist, dass ich meinen Start mit dem Minimum Viable Product (MVP) zu spät begonnen habe, was zu erheblichen Verzögerungen in meiner Planung geführt hat. Für ein Messenger-MVP sollten grundlegende Funktionen wie das Senden und Empfangen von Nachrichten sowie das Starten eines Chats implementiert sein. Erst nachdem diese Funktionen umgesetzt sind, werde ich mich dem UI/UX-Design widmen. Ich erkenne nun den Wert dieser Vorgehensweise und werde sie bei zukünftigen Projekten, sei es privat, akademisch oder beruflich, berücksichtigen. Ausserdem habe ich erkannt, dass ich mein Programm nicht von Grund auf neu erstellen sollte. Hätte ich zuerst mein MVP erreicht, hätte ich anschliessend einfache Funktionen wie Verschlüsselung problemlos hinzufügen können. Leider habe ich in diesem Projekt den umgekehrten Ansatz gewählt.
\subsection{Zukunft des Projekts}
Das Vorhaben besteht darin, kontinuierlich an der Optimierung des Messengers zu arbeiten, sowohl durch Hinzufügen neuer Funktionalitäten als auch durch die Optimierung der bestehenden Codebasis. Dabei wird das primäre Ziel verfolgt, eine konstante Verbesserung und Weiterentwicklung des Projekts zu gewährleisten. Der Quellcode und zukünftige Aktualisierungen sind im zugehörigen Github-Repository unter dem folgenden Link zugänglich:

\href{https://github.com/benni347/messenger}{https://github.com/benni347/messenger}.
\subsection{Danksagung und Mitwirkung}
Mein Dank gilt meinem Vater, der mir insbesondere bei der Erstellung dieser Dokumentation assistierte, indem er die Textlänge überprüfte und Vorschläge für Modifikationen unterbreitete, die den Text verständlicher gestalten könnten.